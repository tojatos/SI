\documentclass[12pt,a4paper]{article}

\usepackage[MeX]{polski}
\usepackage[utf8]{inputenc}
\usepackage[T1]{fontenc}
\usepackage{fullpage}
\usepackage{textcomp}
\usepackage{amsmath, amssymb}
\usepackage{minted}
\usepackage{titling}
\usepackage{graphicx}
\usepackage{mathtools}
\usepackage{hyperref} % clickable table of contents and urls
\usepackage{float}
\usepackage{enumerate} % custom enumeration types
\usepackage[labelfont=bf]{caption} % bold captions
\usepackage{booktabs} % for \midrule
\usepackage[left=1.5cm, right=1.5cm, top=1.5cm, bottom=1.5cm]{geometry} % decrease margins

 % center headings
\usepackage{sectsty}
\allsectionsfont{\centering}

\setlength\parindent{0pt}
\renewcommand*{\thefootnote}{(\arabic{footnote})} % puts footnote in brackets

% minted color box
\usepackage[breakable]{tcolorbox}

\BeforeBeginEnvironment{minted}{\begin{tcolorbox}[breakable]}
\AfterEndEnvironment{minted}{\end{tcolorbox}}

% Create a new environment for breaking code listings across pages.
\usepackage{caption}
\newenvironment{longlisting}{\captionsetup{type=listing}}{}

% break long urls (https://tex.stackexchange.com/a/325037)
\usepackage{etoolbox}
\appto\UrlBreaks{\do\-}

\newcommand{\subtitle}[1]{%
  \posttitle{%
    \par\end{center}
    \begin{center}\large#1\end{center}
    \vskip0.5em}%
}

\newcommand{\nt}[1]{\mkern1mu\overline{\mkern-1mu#1\mkern-1mu}\mkern1mu}
\newcommand{\image}[2] {
    \begin{figure}[H]
        \begin{center}
            \includegraphics[width=0.75\textwidth]{#1}
        \end{center}
        \caption{#2}
        \label{#1}
    \end{figure}
}

% disable numbering
\setcounter{secnumdepth}{0}

\author{Krzysztof Ruczkowski}
\title{Sztuczna Inteligencja i Inżynieria Wiedzy}
\subtitle{Problem spełniania ograniczeń - sprawozdanie}

\begin{document}

\maketitle
\tableofcontents
\newpage

\section{Zbadanie wpływu zastosowania heurystyk na liczbę przeszukanych stanów}

\subsection{Cel badania}
Celem badania jest zbadanie wpływu zastosowania heurystyk na liczbę przeszukanych stanów.
\subsection{Parametry badania}
Badanie przeprowadzone jest na problemie kolorowania mapy dla 4 kolorów.
\subsection{Wyniki i wykresy}
% Table generated by Excel2LaTeX from sheet 'Arkusz1'
\begin{table}[htbp]
  \centering
    \begin{tabular}{lrrrrrrrrrr}
    \multicolumn{11}{c}{\textbf{Forward checking - default variable selector}} \\
    Nodes & 2     & 3     & 4     & 6     & 8     & 9     & 10    & 12    & 13    & 14 \\
\cmidrule{2-11}    First solution nodes & 16    & 40    & 88    & 319   & 1150  & 996   & 1870  & 1360  & 1330  & 1479 \\
    First solution time & 0,04  & 0,01  & 0,02  & 0,01  & 0,02  & 0,02  & 0,02  & 0,01  & 0,02  & 0,02 \\
    All solutions nodes & 16    & 40    & 88    & 1596  & 1228  & 3260  & 3796  & 171956 & 56884 & 498324 \\
    All solutions time & 0,06  & 0,03  & 0,03  & 0,05  & 0,03  & 0,05  & 0,05  & 1,62  & 0,46  & 5,69 \\
    \end{tabular}%
  \caption{Pomiary domyślnego wybierania zmiennej}
  \label{tab:addlabel}%
\end{table}%
% Table generated by Excel2LaTeX from sheet 'Arkusz1'
\begin{table}[htbp]
  \centering
    \begin{tabular}{lrrrrrrrrrr}
    \multicolumn{11}{c}{\textbf{Forward checking - min domain variable selector}} \\
    Nodes & 2     & 3     & 4     & 6     & 8     & 9     & 10    & 12    & 13    & 14 \\
\cmidrule{2-11}    First solution nodes & 16    & 40    & 88    & 976   & 856   & 1840  & 2037  & 1703  & 1341  & 1330 \\
    First solution time & 0,02  & 0,02  & 0,02  & 0,02  & 0,02  & 0,02  & 0,02  & 0,02  & 0,02  & 0,01 \\
    All solutions nodes & 16    & 40    & 88    & 976   & 856   & 1840  & 2584  & 163840 & 12376 & 339064 \\
    All solutions time & 0,03  & 0,03  & 0,03  & 0,03  & 0,03  & 0,03  & 0,03  & 1,41  & 0,16  & 4,18 \\
    \end{tabular}%
  \caption{Pomiary wybierania zmiennej po najmniejszej dziedzinie}
  \label{tab:addlabel}%
\end{table}%
% Table generated by Excel2LaTeX from sheet 'Arkusz1'
\begin{table}[htbp]
  \centering
    \begin{tabular}{lrrrrrrrrrr}
    \multicolumn{11}{c}{\textbf{Forward checking - max domain variable selector}} \\
    Nodes & 2     & 3     & 4     & 6     & 8     & 9     & 10    & 12    & 13    & 14 \\
\cmidrule{2-11}    First solution nodes & 18    & 58    & 203   & 3989  & 4442  & 4831  & 5268  & 4099  & 3441  & 3601 \\
    First solution time & 0,01  & 0,02  & 0,02  & 0,02  & 0,02  & 0,02  & 0,01  & 0,02  & 0,02  & 0,01 \\
    All solutions nodes & 20    & 72    & 253   & 4905  & 25700 & 98730 & 268306 & 7239540 & 7228954 & 46399187 \\
    All solutions time & 0,03  & 0,03  & 0,03  & 0,03  & 0,08  & 0,24  & 0,66  & 20,68 & 20,89 & 141,36 \\
    \end{tabular}%
  \caption{Pomiary wybierania zmiennej po największej dziedzinie}
  \label{tab:addlabel}%
\end{table}%


\image{Arkusz1_3}{Porównanie liczby odwiedzonych węzłów}
\image{Arkusz1_4}{Porównanie czasu wyszukiwania wszystkich rozwiązań}
\subsection{Wnioski}
Heurystyka wyboru zmiennej, która ma najmniej możliwych dostępnych wartości do przybrania, okazała się nieznacznie lepsza od domyślnego wybierania.

Heurystyka wyboru zmiennej, która ma najwięcej możliwych dostępnych wartości do przybrania, okazała się dużo gorsza od domyślnego wybierania.


\newpage
\section{Porównanie metod sprawdzania wprzód oraz przeszukiwania z powrotami}

\subsection{Cel badania}
Celem badania jest porównanie metod sprawdzania wprzód oraz przeszukiwania z powrotami.

\subsection{Parametry badania}
Badanie przeprowadzone jest na problemach Einstein'a\footnote{\url{https://en.wikipedia.org/wiki/Zebra_Puzzle}} i problemie kolorowania map dla 4 kolorów.


Użyta została heurystyka wyboru zmiennej, która ma najmniej możliwych dostępnych wartości do przybrania.
% \newpage
\subsection{Wyniki i wykresy}
% Table generated by Excel2LaTeX from sheet 'Arkusz1'
\begin{table}[htbp]
  \centering
    \begin{tabular}{lrr}
    \multicolumn{3}{c}{\textbf{Einstein CSP}} \\
          &       &  \\
          & \multicolumn{1}{c}{\textbf{Backtracking}} & \multicolumn{1}{c}{\textbf{Forward checking}} \\
    First solution nodes & 2050  & 80 \\
    First solution time & 0,02  & 0,15 \\
    All solutions nodes & 4105  & 193 \\
    All solutions time & 0,04  & 0,19 \\
    \end{tabular}%
  \caption{Wyniki badania dla problemu Einstein'a}
  \label{tab:addlabel}%
\end{table}%

% Table generated by Excel2LaTeX from sheet 'Arkusz1'
\begin{table}[htbp]
  \centering
    \begin{tabular}{lrrrrrrrrrr}
    \multicolumn{11}{c}{\textbf{Map Coloring CSP}} \\
          &       &       &       &       &       &       &       &       &       &  \\
    \multicolumn{11}{c}{\textbf{Backtracking}} \\
    Nodes & 2     & 3     & 4     & 6     & 8     & 9     & 10    & 12    & 13    & 14 \\
\cmidrule{2-11}    First solution nodes & 19    & 65    & 227   & 1422  & 1884  & 7430  & 5890  & 5252  & 4848  & 7084 \\
    First solution time & 0,03  & 0,02  & 0,02  & 0,02  & 0,02  & 0,02  & 0,02  & 0,02  & 0,02  & 0,02 \\
    All solutions nodes & 20    & 68    & 228   & 1428  & 1892  & 11508 & 27284 & 184596 & 534932 & 1550548 \\
    All solutions time & 0,05  & 0,04  & 0,04  & 0,04  & 0,04  & 0,04  & 0,08  & 0,74  & 2,21  & 5,34 \\
          &       &       &       &       &       &       &       &       &       &  \\
          &       &       &       &       &       &       &       &       &       &  \\
          &       &       &       &       &       &       &       &       &       &  \\
    \multicolumn{11}{c}{\textbf{Forward checking}} \\
    Nodes & 2     & 3     & 4     & 6     & 8     & 9     & 10    & 12    & 13    & 14 \\
\cmidrule{2-11}    First solution nodes & 16    & 40    & 88    & 556   & 2007  & 2645  & 2728  & 2108  & 2117  & 1637 \\
    First solution time & 0,05  & 0,02  & 0,02  & 0,02  & 0,02  & 0,02  & 0,02  & 0,02  & 0,02  & 0,02 \\
    All solutions nodes & 16    & 40    & 88    & 664   & 3928  & 7864  & 2728  & 43000 & 52984 & 61576 \\
    All solutions time & 0,06  & 0,04  & 0,04  & 0,04  & 0,06  & 0,06  & 0,04  & 0,42  & 0,52  & 0,80 \\
    \end{tabular}%
  \caption{Wyniki badania dla problemu kolorowania mapy}
  \label{tab:addlabel}%
\end{table}%



\image{Arkusz1_1}{Porównanie liczby odwiedzonych węzłów}
\image{Arkusz1_2}{Porównanie czasu wyszukiwania wszystkich rozwiązań}
\subsection{Wnioski}
Wyszukiwanie w przód pozwala na znaczącą oszczędność czasu dla bardziej skomplikowanych problemów.

% Table generated by Excel2LaTeX from sheet 'Porównanie algorytmów'
% \begin{table}[htbp]
%   \centering
%     \begin{tabular}{|rrrr}
%     \multicolumn{4}{|c}{\textbf{Metoda losowych mutacji [30 x 1500]}} \\
%     \multicolumn{1}{|c}{\textbf{best}} & \multicolumn{1}{c}{\textbf{worst}} & \multicolumn{1}{c}{\textbf{avg}} & \multicolumn{1}{c}{\textbf{std}} \\
%     \midrule
%     -349442 & -604234 & -470676 & 57255,29 \\
%     -233683 & -420265 & -330202 & 50262,56 \\
%     -637791 & -935288 & -806516 & 79839,92 \\
%     \end{tabular}%
%   \caption{Porównanie algorytmu genetycznego z naiwnymi}
%   \label{tab:addlabel}%
% \end{table}%



% \image{images/Porównanie algorytmów_1}{Porównanie algorytmów - problem 1}
% \image{images/Porównanie algorytmów_2}{Porównanie algorytmów - problem 2}
% \image{images/Porównanie algorytmów_3}{Porównanie algorytmów - problem 3}

\newpage
\section{Podsumowanie}
Heurystyka wybierania zmiennej, która ma najmniej możliwych dostępnych wartości do przybrania, pozwala na niewielkie usprawnienie wyszukiwania rozwiązań.
Metoda sprawdzania wprzód działa dużo szybciej i przeszukuje mniej stanów w porównaniu do metody przeszukiwania z powrotami.

% \image{images/solution1}{Najlepsze znalezione rozwiązanie do problemu 1}
% \image{images/solution2}{Najlepsze znalezione rozwiązanie do problemu 2}
% \image{images/solution3}{Najlepsze znalezione rozwiązanie do problemu 3}

\end{document}
