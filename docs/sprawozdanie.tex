\documentclass[12pt,a4paper]{article}

\usepackage[MeX]{polski}
\usepackage[utf8]{inputenc}
\usepackage[T1]{fontenc}
\usepackage{fullpage}
\usepackage{textcomp}
\usepackage{amsmath, amssymb}
\usepackage{minted}
\usepackage{titling}
\usepackage{graphicx}
\usepackage{mathtools}
\usepackage{hyperref} % clickable table of contents and urls
\usepackage{float}
\usepackage{enumerate} % custom enumeration types
\usepackage[labelfont=bf]{caption} % bold captions
\usepackage{booktabs} % for \midrule

 % center headings
\usepackage{sectsty}
\allsectionsfont{\centering}

\setlength\parindent{0pt}
\renewcommand*{\thefootnote}{(\arabic{footnote})} % puts footnote in brackets

% minted color box
\usepackage[breakable]{tcolorbox}

\BeforeBeginEnvironment{minted}{\begin{tcolorbox}[breakable]}
\AfterEndEnvironment{minted}{\end{tcolorbox}}

% Create a new environment for breaking code listings across pages.
\usepackage{caption}
\newenvironment{longlisting}{\captionsetup{type=listing}}{}

% break long urls (https://tex.stackexchange.com/a/325037)
\usepackage{etoolbox}
\appto\UrlBreaks{\do\-}

\newcommand{\subtitle}[1]{%
  \posttitle{%
    \par\end{center}
    \begin{center}\large#1\end{center}
    \vskip0.5em}%
}

\newcommand{\nt}[1]{\mkern1mu\overline{\mkern-1mu#1\mkern-1mu}\mkern1mu}
\newcommand{\image}[2] {
    \begin{figure}[H]
        \begin{center}
            \includegraphics[width=0.75\textwidth]{#1}
        \end{center}
        \caption{#2}
        \label{#1}
    \end{figure}
}

% disable numbering
\setcounter{secnumdepth}{0}

\author{Krzysztof Ruczkowski}
\title{Sztuczna Inteligencja i Inżynieria Wiedzy}
\subtitle{Algorytmy Genetyczne - sprawozdanie}

\begin{document}

\maketitle
\tableofcontents
\newpage

\section{Badania wpływu parametrów}
Każde badanie jest uruchamiane 5 razy.
\subsection{Rozmiar populacji}
\subsubsection{Cel badania}
Celem badania jest wybranie rozmiaru populacji, dla którego algorytm genetyczny daje najlepsze wyniki.
\subsubsection{Wyniki i wykresy}
% Table generated by Excel2LaTeX from sheet 'Badanie rozmiaru populacji'
\begin{table}[htbp]
  \centering
    \begin{tabular}{crrrlrrrrrr}
    \multicolumn{1}{p{3.715em}}{\textbf{Problem}} & \multicolumn{1}{p{5.93em}}{\textbf{Rozmiar populacji}} & \multicolumn{1}{p{2.645em}}{\textbf{best}} & \multicolumn{1}{p{2.93em}}{\textbf{worst}} & \multicolumn{1}{p{3.785em}}{\textbf{avg}} & \multicolumn{1}{p{3.5em}}{\textbf{std}} \\
    \midrule
    zad1  & 10     & -328  & -1034 & -613,60 & 333,62 \\
    zad1  & 100    & -328  & -1039 & -745,20 & 335,94 \\
    zad1  & 500    & -342  & -1031 & -888,80 & 273,44 \\
    zad1  & 1000   & -349  & -1027 & -883,80 & 267,54 \\
    zad1  & 2000   & -343  & -1028 & -750,40 & 331,11 \\
    zad1  & 5000   & -319  & -1021 & -873,60 & 277,34 \\
    zad1  & 10000  & -319  & -1027 & -613,80 & 331,19 \\
    \midrule
    zad2  & 10     & -424  & -430  & -425,20 & 2,40 \\
    zad2  & 100    & -424  & -430  & -427,00 & 2,53 \\
    zad2  & 500    & -424  & -1036 & -548,20 & 243,92 \\
    zad2  & 1000   & -424  & -430  & -425,20 & 2,40 \\
    zad2  & 2000   & -424  & -450  & -429,20 & 10,40 \\
    zad2  & 5000   & -424  & -1036 & -546,40 & 244,80 \\
    zad2  & 10000  & -424  & -1036 & -546,40 & 244,80 \\
    \midrule
    zad3  & 10     & -2071 & -3517 & -2882,00 & 526,57 \\
    zad3  & 100    & -2097 & -4252 & -2794,80 & 785,06 \\
    zad3  & 500    & -1436 & -3464 & -2527,80 & 678,10 \\
    zad3  & 1000   & -2114 & -3553 & -2975,00 & 530,80 \\
    zad3  & 2000   & -2042 & -3530 & -2779,80 & 476,21 \\
    zad3  & 5000   & -1492 & -3588 & -2838,60 & 761,07 \\
    zad3  & 10000  & -2100 & -2819 & -2392,60 & 307,51 \\
    \end{tabular}%
  \caption{Badanie rozmiaru populacji}
  \label{tab:addlabel}%
\end{table}%

\image{images/Badanie rozmiaru populacji_1}{Badanie rozmiaru populacji - problem 1}
\image{images/Badanie rozmiaru populacji_2}{Badanie rozmiaru populacji - problem 2}
\image{images/Badanie rozmiaru populacji_3}{Badanie rozmiaru populacji - problem 3}
\subsubsection{Wnioski}
Zmiana rozmiaru populacji ma niewielki wpływ na problemy 1 i 2.
Wybrany zostaje rozmiar populacji "500", ponieważ zwraca on najlepsze wyniki dla problemu 3.

\subsection{Liczba pokoleń}
\subsubsection{Cel badania}
Celem badania jest wybranie liczby pokoleń, po której algorytm genetyczny przestaje zwracać lepsze wyniki (lub robi to znacząco wolniej).
\subsubsection{Wyniki i wykresy}
\image{images/Badanie liczby generacji_1}{Badanie liczby pokoleń - problem 1}
\image{images/Badanie liczby generacji_2}{Badanie liczby pokoleń - problem 2}
\image{images/Badanie liczby generacji_3}{Badanie liczby pokoleń - problem 3}
\subsubsection{Wnioski}
Rozwiązania dosyć szybko zbiegają do minimum.
Wybrana zostaje liczba pokoleń "1500", ponieważ uznałem to za dobry kompromis pomiędzy czasem trwania badań a drobnymi korzyściami wyników dla problemu 3.

\subsection{Rozmiar turnieju}
\subsubsection{Cel badania}
Celem badania jest wybranie rozmiaru turnieju, dla którego algorytm genetyczny daje najlepsze wyniki.
\subsubsection{Wyniki i wykresy}
\image{images/Badanie rozmiaru turnieju_1}{Badanie rozmiaru turnieju - problem 1}
\image{images/Badanie rozmiaru turnieju_2}{Badanie rozmiaru turnieju - problem 2}
\image{images/Badanie rozmiaru turnieju_3}{Badanie rozmiaru turnieju - problem 3}
\subsubsection{Wnioski}
Wybrany zostaje rozmiar turnieju "3", ponieważ zwraca on najlepsze wyniki dla wszystkich problemów.

\subsection{Operatory selekcji}
\subsubsection{Cel badania}
Celem badania jest wybranie operatora selekcji, dla którego algorytm genetyczny daje najlepsze wyniki.
\subsubsection{Wyniki i wykresy}
\image{images/Badanie operatorów_1}{Badanie operatorów selekcji - problem 1}
\image{images/Badanie operatorów_2}{Badanie operatorów selekcji - problem 2}
\image{images/Badanie operatorów_3}{Badanie operatorów selekcji - problem 3}
\subsubsection{Wnioski}
Wybór operatora nie ma znaczenia dla dwóch pierwszych problemów.
Wybrany zostaje operator turnieju, ponieważ zwraca on najlepsze wyniki dla problemu 3.

\subsection{Prawdopodobieństwo krzyżowania}
\subsubsection{Cel badania}
Celem badania jest wybranie prawdopodobieństwa krzyżowania, dla którego algorytm genetyczny daje najlepsze wyniki.
\subsubsection{Wyniki i wykresy}
\image{images/Badanie p. krzyżowania_1}{Badanie prawdopodobieństwa krzyżowania - problem 1}
\image{images/Badanie p. krzyżowania_2}{Badanie prawdopodobieństwa krzyżowania - problem 2}
\image{images/Badanie p. krzyżowania_3}{Badanie prawdopodobieństwa krzyżowania - problem 3}
\subsubsection{Wnioski}
Wybór prawdopodobieństwa krzyżowania nie ma znaczenia dla dwóch pierwszych problemów.
Wybrane zostaje prawdopodobieństwo krzyżowania "0.2", ponieważ zwraca ono najlepsze wyniki dla problemu 3.

\subsection{Prawdopodobieństwo mutacji}
\subsubsection{Cel badania}
Celem badania jest wybranie prawdopodobieństwa mutacji, dla którego algorytm genetyczny daje najlepsze wyniki.
\subsubsection{Wyniki i wykresy}
\image{images/Badanie p. mutacji_1}{Badanie prawdopodobieństwa mutacji - problem 1}
\image{images/Badanie p. mutacji_2}{Badanie prawdopodobieństwa mutacji - problem 2}
\image{images/Badanie p. mutacji_3}{Badanie prawdopodobieństwa mutacji - problem 3}
\subsubsection{Wnioski}
Wybrane zostaje prawdopodobieństwo mutacji "0.9", ponieważ zwraca ono najlepsze wyniki dla wszystkich problemów.

\section{Porównanie algorytmu genetycznego z metodami "naiwnymi"}
\image{images/Porównanie algorytmów_1}{Porównanie algorytmów - problem 1}
\image{images/Porównanie algorytmów_2}{Porównanie algorytmów - problem 2}
\image{images/Porównanie algorytmów_3}{Porównanie algorytmów - problem 3}

\section{Podsumowanie}
Algorytm genetyczny pozwala na znaczące poprawy wyników dla danego problemu.
\image{images/solution1}{Najlepsze znalezione rozwiązanie do problemu 1}
\image{images/solution2}{Najlepsze znalezione rozwiązanie do problemu 2}
\image{images/solution3}{Najlepsze znalezione rozwiązanie do problemu 3}

\end{document}
