\documentclass[12pt,a4paper]{article}

\usepackage[MeX]{polski}
\usepackage[utf8]{inputenc}
\usepackage[T1]{fontenc}
\usepackage{fullpage}
\usepackage{textcomp}
\usepackage{amsmath, amssymb}
\usepackage{minted}
\usepackage{titling}
\usepackage{graphicx}
\usepackage{mathtools}
\usepackage{hyperref} % clickable table of contents and urls
\usepackage{float}
\usepackage{enumerate} % custom enumeration types
\usepackage[labelfont=bf]{caption} % bold captions

 % center headings
\usepackage{sectsty}
\allsectionsfont{\centering}

\setlength\parindent{0pt}
\renewcommand*{\thefootnote}{(\arabic{footnote})} % puts footnote in brackets

% minted color box
\usepackage[breakable]{tcolorbox}

\BeforeBeginEnvironment{minted}{\begin{tcolorbox}[breakable]}
\AfterEndEnvironment{minted}{\end{tcolorbox}}

% Create a new environment for breaking code listings across pages.
\usepackage{caption}
\newenvironment{longlisting}{\captionsetup{type=listing}}{}

% break long urls (https://tex.stackexchange.com/a/325037)
\usepackage{etoolbox}
\appto\UrlBreaks{\do\-}

\newcommand{\subtitle}[1]{%
  \posttitle{%
    \par\end{center}
    \begin{center}\large#1\end{center}
    \vskip0.5em}%
}

\newcommand{\nt}[1]{\mkern1mu\overline{\mkern-1mu#1\mkern-1mu}\mkern1mu}
\newcommand{\image}[2] {
    \begin{figure}[H]
        \begin{center}
            \includegraphics[width=0.75\textwidth]{#1}
        \end{center}
        \caption{#2}
        \label{#1}
    \end{figure}
}

% disable numbering
\setcounter{secnumdepth}{0}

\author{Krzysztof Ruczkowski}
\title{Sztuczna Inteligencja i Inżynieria Wiedzy}
\subtitle{Algorytmy Genetyczne - sprawozdanie}

\begin{document}

\maketitle
\tableofcontents
\newpage

\section{Badania wpływu parametrów}
Każde badanie jest uruchamiane 5 razy.
5000 iteracji.
\subsection{Rozmiar populacji}
\subsection{Liczba pokoleń}
\subsection{Rozmiar turnieju}
\subsection{Operatorów selekcji}
\subsection{Prawdopodobieństwo krzyżowania}
\subsection{Prawdopodobieństwo mutacji}
\section{Porównanie algorytmu genetycznego z metodami "naiwnymi"}
\subsection{Metoda losowa}
\subsection{Metoda losowych mutacji}

\section{Podsumowanie}
\image{solution1}{Najlepsze znalezione rozwiązanie do problemu 1}
\image{solution2}{Najlepsze znalezione rozwiązanie do problemu 2}
\image{solution3}{Najlepsze znalezione rozwiązanie do problemu 3}

\end{document}
