\documentclass[12pt,a4paper]{article}

\usepackage[MeX]{polski}
\usepackage[utf8]{inputenc}
\usepackage[T1]{fontenc}
\usepackage{fullpage}
\usepackage{textcomp}
\usepackage{amsmath, amssymb}
\usepackage{minted}
\usepackage{titling}
\usepackage{graphicx}
\usepackage{mathtools}
\usepackage{hyperref} % clickable table of contents and urls
\usepackage{float}
\usepackage{enumerate} % custom enumeration types
\usepackage[labelfont=bf]{caption} % bold captions
\usepackage{booktabs} % for \midrule

 % center headings
\usepackage{sectsty}
\allsectionsfont{\centering}

\setlength\parindent{0pt}
\renewcommand*{\thefootnote}{(\arabic{footnote})} % puts footnote in brackets

% minted color box
\usepackage[breakable]{tcolorbox}

\BeforeBeginEnvironment{minted}{\begin{tcolorbox}[breakable]}
\AfterEndEnvironment{minted}{\end{tcolorbox}}

% Create a new environment for breaking code listings across pages.
\usepackage{caption}
\newenvironment{longlisting}{\captionsetup{type=listing}}{}

% break long urls (https://tex.stackexchange.com/a/325037)
\usepackage{etoolbox}
\appto\UrlBreaks{\do\-}

\newcommand{\subtitle}[1]{%
  \posttitle{%
    \par\end{center}
    \begin{center}\large#1\end{center}
    \vskip0.5em}%
}

\newcommand{\nt}[1]{\mkern1mu\overline{\mkern-1mu#1\mkern-1mu}\mkern1mu}
\newcommand{\image}[2] {
    \begin{figure}[H]
        \begin{center}
            \includegraphics[width=0.75\textwidth]{#1}
        \end{center}
        \caption{#2}
        \label{#1}
    \end{figure}
}

% disable numbering
\setcounter{secnumdepth}{0}

\author{Krzysztof Ruczkowski}
\title{Sztuczna Inteligencja i Inżynieria Wiedzy}
\subtitle{Algorytmy Genetyczne - sprawozdanie}

\begin{document}

\maketitle
\tableofcontents
\newpage

\section{Badania wpływu parametrów}
W tym przypadku algorytmu genetycznego dążę do maksymalizacji funkcji przystosowania.

Pomijając badanie liczby pokoleń:
Każde badanie jest uruchamiane 5 razy.
Każde uruchomienie zwraca wynik najlepszego znalezionego osobnika.
Wartość "best" to najlepszy wynik z tych 5 uruchomień, a "worst" to najgorszy.
Wartość "avg" to średni wynik, a "std" to odchylenie standardowe wyników.
Na wykresach oś X przedstawia badany parametr, a oś Y wartość wyniku / odchylenie.
\subsection{Rozmiar populacji}
\subsubsection{Cel badania}
Celem badania jest wybranie rozmiaru populacji, dla którego algorytm genetyczny daje najlepsze wyniki.

\subsubsection{Stałe parametry badania}
% Table generated by Excel2LaTeX from sheet 'Badanie rozmiaru populacji'
\begin{table}[htbp]
  \centering
    \begin{tabular}{rrlrr}
    \multicolumn{1}{p{3.93em}}{\textbf{Liczba pokoleń}} & \multicolumn{1}{p{4.07em}}{\textbf{Rozmiar turnieju}} & \multicolumn{1}{p{4.07em}}{\textbf{Operator selekcji}} & \multicolumn{1}{p{5.715em}}{\textbf{P. krzyżowania}} & \multicolumn{1}{p{5.355em}}{\textbf{P.  mutacji}} \\
    \midrule
    1000  & 3     & Turniej & 0,1   & 0,8 \\
    \end{tabular}%
  \label{tab:addlabel}%
  \caption{Stałe parametry badania}
\end{table}%
\subsubsection{Wyniki i wykresy}
% Table generated by Excel2LaTeX from sheet 'Badanie rozmiaru populacji'
\begin{table}[htbp]
  \centering
    \begin{tabular}{crrrlrrrrrr}
    \multicolumn{1}{p{3.715em}}{\textbf{Problem}} & \multicolumn{1}{p{5.93em}}{\textbf{Rozmiar populacji}} & \multicolumn{1}{p{2.645em}}{\textbf{best}} & \multicolumn{1}{p{2.93em}}{\textbf{worst}} & \multicolumn{1}{p{3.785em}}{\textbf{avg}} & \multicolumn{1}{p{3.5em}}{\textbf{std}} \\
    \midrule
    zad1  & 10     & -328  & -1034 & -613,60 & 333,62 \\
    zad1  & 100    & -328  & -1039 & -745,20 & 335,94 \\
    zad1  & 500    & -342  & -1031 & -888,80 & 273,44 \\
    zad1  & 1000   & -349  & -1027 & -883,80 & 267,54 \\
    zad1  & 2000   & -343  & -1028 & -750,40 & 331,11 \\
    zad1  & 5000   & -319  & -1021 & -873,60 & 277,34 \\
    zad1  & 10000  & -319  & -1027 & -613,80 & 331,19 \\
    \midrule
    zad2  & 10     & -424  & -430  & -425,20 & 2,40 \\
    zad2  & 100    & -424  & -430  & -427,00 & 2,53 \\
    zad2  & 500    & -424  & -1036 & -548,20 & 243,92 \\
    zad2  & 1000   & -424  & -430  & -425,20 & 2,40 \\
    zad2  & 2000   & -424  & -450  & -429,20 & 10,40 \\
    zad2  & 5000   & -424  & -1036 & -546,40 & 244,80 \\
    zad2  & 10000  & -424  & -1036 & -546,40 & 244,80 \\
    \midrule
    zad3  & 10     & -2071 & -3517 & -2882,00 & 526,57 \\
    zad3  & 100    & -2097 & -4252 & -2794,80 & 785,06 \\
    zad3  & 500    & -1436 & -3464 & -2527,80 & 678,10 \\
    zad3  & 1000   & -2114 & -3553 & -2975,00 & 530,80 \\
    zad3  & 2000   & -2042 & -3530 & -2779,80 & 476,21 \\
    zad3  & 5000   & -1492 & -3588 & -2838,60 & 761,07 \\
    zad3  & 10000  & -2100 & -2819 & -2392,60 & 307,51 \\
    \end{tabular}%
  \caption{Badanie rozmiaru populacji}
  \label{tab:addlabel}%
\end{table}%

\image{images/Badanie rozmiaru populacji_1}{Badanie rozmiaru populacji - problem 1}
\image{images/Badanie rozmiaru populacji_2}{Badanie rozmiaru populacji - problem 2}
\image{images/Badanie rozmiaru populacji_3}{Badanie rozmiaru populacji - problem 3}
\subsubsection{Wnioski}
Zmiana rozmiaru populacji ma niewielki wpływ na problemy 1 i 2.
Wybrany zostaje rozmiar populacji "500", ponieważ zwraca on najlepsze wyniki dla problemu 3.

\subsection{Liczba pokoleń}
Badanie polega na uruchomieniu algorytmu genetycznego i zapisanie najlepszego, najgorszego i średniego wyniku w populacji danej generacji, oraz odchylenie standardowe wyników populacji.
Na wykresie oś X przedstawia numer generacji, a oś Y wartość wyniku / odchylenie.

\subsubsection{Cel badania}
Celem badania jest wybranie liczby pokoleń, po której algorytm genetyczny przestaje zwracać lepsze wyniki (lub robi to znacząco wolniej).
\subsubsection{Stałe parametry badania}
% Table generated by Excel2LaTeX from sheet 'Badanie liczby generacji'
\begin{table}[htbp]
  \centering
    \begin{tabular}{rrrlrr}
    \multicolumn{1}{p{4.215em}}{\textbf{Rozmiar populacji}} & \multicolumn{1}{p{4.215em}}{\textbf{Liczba pokoleń}} & \multicolumn{1}{p{4.215em}}{\textbf{Rozmiar turnieju}} & \multicolumn{1}{p{4.215em}}{\textbf{Operator selekcji}} & \multicolumn{1}{p{5.5em}}{\textbf{Pr. krzyżowania}} & \multicolumn{1}{p{4.215em}}{\textbf{Pr.  mutacji}} \\
    \midrule
    500   & 1000  & 3     & Turniej & 0,1   & 0,8 \\
    \end{tabular}%
  \label{tab:addlabel}%
  \caption{Stałe parametry badania}
\end{table}%

\subsubsection{Wyniki i wykresy}
\image{images/Badanie liczby generacji_1}{Badanie liczby pokoleń - problem 1}
\image{images/Badanie liczby generacji_2}{Badanie liczby pokoleń - problem 2}
\image{images/Badanie liczby generacji_3}{Badanie liczby pokoleń - problem 3}
\subsubsection{Wnioski}
Rozwiązania dosyć szybko zbiegają do minimum.
Wybrana zostaje liczba pokoleń "1500", ponieważ uznałem to za dobry kompromis pomiędzy czasem trwania badań a drobnymi korzyściami wyników dla problemu 3.

\subsection{Rozmiar turnieju}
\subsubsection{Cel badania}
Celem badania jest wybranie rozmiaru turnieju, dla którego algorytm genetyczny daje najlepsze wyniki.
\subsubsection{Stałe parametry badania}
% Table generated by Excel2LaTeX from sheet 'Badanie rozmiaru turnieju'
\begin{table}[htbp]
  \centering
    \begin{tabular}{rrrr}
    \multicolumn{1}{p{8.715em}}{\textbf{Rozmiar populacji}} & \multicolumn{1}{p{7.43em}}{\textbf{Liczba pokoleń}} & \multicolumn{1}{p{9.145em}}{\textbf{Pr. krzyżowania}} & \multicolumn{1}{p{8.355em}}{\textbf{Pr. mutacji}} \\
    \midrule
    500   & 1500  & 0,1   & 0,8 \\
    \end{tabular}%
  \caption{Stałe parametry badania}
  \label{tab:addlabel}%
\end{table}%

\subsubsection{Wyniki i wykresy}
% Table generated by Excel2LaTeX from sheet 'Badanie rozmiaru turnieju'
\begin{table}[htbp]
  \centering
    \begin{tabular}{crrrrr}
    \multicolumn{1}{p{4.855em}}{\textbf{Problem}} & \multicolumn{1}{p{7.715em}}{\textbf{Rozmiar turnieju}} & \multicolumn{1}{p{4.785em}}{\textbf{best}} & \multicolumn{1}{p{4.785em}}{\textbf{worst}} & \multicolumn{1}{p{5em}}{\textbf{avg}} & \multicolumn{1}{p{3.93em}}{\textbf{std}} \\
    \midrule
    zad1  & 1     & -29318 & -36382 & -32366,20 & 2593,06 \\
    zad1  & 2     & -348  & -1754 & -1052,60 & 444,99 \\
    zad1  & 3     & -359  & -1030 & -894,00 & 267,50 \\
    zad1  & 5     & -319  & -1027 & -466,60 & 280,31 \\
    zad1  & 10    & -319  & -1043 & -751,20 & 335,38 \\
    zad1  & 20    & -328  & -1853 & -780,80 & 599,69 \\
    zad1  & 100   & -342  & -1852 & -939,80 & 558,65 \\
    \midrule
    zad2  & 1     & -6128 & -8437 & -7273,80 & 783,57 \\
    zad2  & 2     & -1036 & -1036 & -1036,00 & 0,00 \\
    zad2  & 3     & -424  & -1036 & -546,40 & 244,80 \\
    zad2  & 5     & -424  & -424  & -424,00 & 0,00 \\
    zad2  & 10    & -424  & -424  & -424,00 & 0,00 \\
    zad2  & 20    & -424  & -1100 & -694,40 & 331,17 \\
    zad2  & 100   & -424  & -1100 & -559,20 & 270,40 \\
    \midrule
    zad3  & 1     & -70506 & -149770 & -113927,40 & 28986,10 \\
    zad3  & 2     & -2856 & -13525 & -6397,80 & 3713,59 \\
    zad3  & 3     & -1520 & -4249 & -2595,00 & 924,15 \\
    zad3  & 5     & -1426 & -2800 & -1993,80 & 493,99 \\
    zad3  & 10    & -2845 & -4241 & -3787,60 & 552,69 \\
    zad3  & 20    & -2146 & -4334 & -3255,20 & 874,21 \\
    zad3  & 100   & -2787 & -5536 & -3860,20 & 939,94 \\
    \end{tabular}%
  \label{tab:addlabel}%
  \caption{Badanie rozmiaru turnieju}
\end{table}%

\image{images/Badanie rozmiaru turnieju_1}{Badanie rozmiaru turnieju - problem 1}
\image{images/Badanie rozmiaru turnieju_2}{Badanie rozmiaru turnieju - problem 2}
\image{images/Badanie rozmiaru turnieju_3}{Badanie rozmiaru turnieju - problem 3}
\subsubsection{Wnioski}
Wybrany zostaje rozmiar turnieju "3", ponieważ zwraca on najlepsze wyniki dla wszystkich problemów.

\subsection{Operatory selekcji}
\subsubsection{Cel badania}
Celem badania jest wybranie operatora selekcji, dla którego algorytm genetyczny daje najlepsze wyniki.

\subsubsection{Stałe parametry badania}
% Table generated by Excel2LaTeX from sheet 'Badanie operatorów'
\begin{table}[htbp]
  \centering
    \begin{tabular}{rrrrr}
    \multicolumn{1}{p{4.215em}}{\textbf{Rozmiar populacji}} & \multicolumn{1}{p{4.215em}}{\textbf{Liczba pokoleń}} & \multicolumn{1}{p{4.215em}}{\textbf{Rozmiar turnieju}} & \multicolumn{1}{p{6.145em}}{\textbf{Pr. krzyżowania}} & \multicolumn{1}{p{4.215em}}{\textbf{Pr. mutacji}} \\
    \midrule
    500   & 1500  & 3     & 0,1   & 0,8 \\
    \end{tabular}%
  \caption{Stałe parametry badania}
  \label{tab:addlabel}%
\end{table}%

\subsubsection{Wyniki i wykresy}
% Table generated by Excel2LaTeX from sheet 'Badanie operatorów'
\begin{table}[htbp]
  \centering
    \begin{tabular}{clrrrr}
    \multicolumn{1}{p{4.215em}}{\textbf{Problem}} & \multicolumn{1}{p{4.215em}}{\textbf{Operator selekcji}} & \multicolumn{1}{p{4.215em}}{\textbf{best}} & \multicolumn{1}{p{4.215em}}{\textbf{worst}} & \multicolumn{1}{p{4.215em}}{\textbf{avg}} & \multicolumn{1}{p{4.215em}}{\textbf{std}} \\
    \midrule
    zad1  & Ruletka & -340  & -1047 & -635,20 & 330,02 \\
    zad1  & Turniej & -341  & -1833 & -1052,40 & 472,79 \\
    \midrule
    zad2  & Ruletka & -424  & -1036 & -546,40 & 244,80 \\
    zad2  & Turniej & -424  & -1036 & -547,60 & 244,21 \\
    \midrule
    zad3  & Ruletka & -3486 & -15613 & -10802,60 & 5146,92 \\
    zad3  & Turniej & -2018 & -4351 & -2947,60 & 882,01 \\
    \end{tabular}%
  \label{tab:addlabel}%
  \caption{Badanie operatorów selekcji}
\end{table}%

\image{images/Badanie operatorów_1}{Badanie operatorów selekcji - problem 1}
\image{images/Badanie operatorów_2}{Badanie operatorów selekcji - problem 2}
\image{images/Badanie operatorów_3}{Badanie operatorów selekcji - problem 3}
\subsubsection{Wnioski}
Wybór operatora nie ma znaczenia dla dwóch pierwszych problemów.
Wybrany zostaje operator turnieju, ponieważ zwraca on najlepsze wyniki dla problemu 3.

\subsection{Prawdopodobieństwo krzyżowania}
\subsubsection{Cel badania}
Celem badania jest wybranie prawdopodobieństwa krzyżowania, dla którego algorytm genetyczny daje najlepsze wyniki.

\subsubsection{Stałe parametry badania}
% Table generated by Excel2LaTeX from sheet 'Badanie p. krzyżowania'
\begin{table}[htbp]
  \centering
    \begin{tabular}{rrrlr}
    \multicolumn{1}{p{4.215em}}{\textbf{Rozmiar populacji}} & \multicolumn{1}{p{4.215em}}{\textbf{Liczba pokoleń}} & \multicolumn{1}{p{4.215em}}{\textbf{Rozmiar turnieju}} & \multicolumn{1}{p{4.215em}}{\textbf{Operator selekcji}} & \multicolumn{1}{p{4.215em}}{\textbf{Pr. mutacji}} \\
    \midrule
    500   & 1500  & 3     & Turniej & 0,8 \\
    \end{tabular}%
  \caption{Stałe parametry badania}
  \label{tab:addlabel}%
\end{table}%

\subsubsection{Wyniki i wykresy}
% Table generated by Excel2LaTeX from sheet 'Badanie p. krzyżowania'
\begin{table}[htbp]
  \centering
    \begin{tabular}{crrrrr}
    \multicolumn{1}{p{4.215em}}{\textbf{Problem}} & \multicolumn{1}{p{6.645em}}{\textbf{Pr. krzyżowania}} & \multicolumn{1}{p{4.215em}}{\textbf{best}} & \multicolumn{1}{p{4.215em}}{\textbf{worst}} & \multicolumn{1}{p{4.215em}}{\textbf{avg}} & \multicolumn{1}{p{4.215em}}{\textbf{std}} \\
    \midrule
    zad1  & 0,01  & -328  & -1055 & -895,60 & 284,28 \\
    zad1  & 0,05  & -326  & -1047 & -486,20 & 281,10 \\
    zad1  & 0,1   & -347  & -1042 & -510,40 & 267,20 \\
    zad1  & 0,2   & -328  & -1048 & -493,60 & 278,44 \\
    zad1  & 0,3   & -319  & -349  & -334,80 & 10,34 \\
    zad1  & 0,5   & -328  & -1024 & -476,00 & 274,24 \\
    zad1  & 0,8   & -320  & -1027 & -606,80 & 342,36 \\
    zad1  & 0,9   & -319  & -1042 & -610,60 & 342,24 \\
    zad1  & 1     & -328  & -363  & -343,60 & 14,92 \\
    \midrule
    zad2  & 0,01  & -424  & -1036 & -546,40 & 244,80 \\
    zad2  & 0,05  & -424  & -1036 & -546,40 & 244,80 \\
    zad2  & 0,1   & -424  & -1036 & -546,40 & 244,80 \\
    zad2  & 0,2   & -424  & -424  & -424,00 & 0,00 \\
    zad2  & 0,3   & -424  & -1036 & -546,40 & 244,80 \\
    zad2  & 0,5   & -424  & -436  & -427,80 & 4,66 \\
    zad2  & 0,8   & -430  & -438  & -432,00 & 3,03 \\
    zad2  & 0,9   & -424  & -430  & -425,60 & 2,24 \\
    zad2  & 1     & -424  & -1036 & -548,80 & 243,61 \\
    \midrule
    zad3  & 0,01  & -2116 & -5606 & -3262,4 & 1283,987 \\
    zad3  & 0,05  & -2714 & -5665 & -3365,6 & 1150,938 \\
    zad3  & 0,1   & -2110 & -4193 & -2890,60 & 692,32 \\
    zad3  & 0,2   & -706  & -4305 & -2399,80 & 1160,28 \\
    zad3  & 0,3   & -1512 & -4818 & -3076,20 & 1079,32 \\
    zad3  & 0,5   & -811  & -2925 & -1971,8 & 674,2876 \\
    zad3  & 0,8   & -743  & -2116 & -1820,2 & 539,8316 \\
    zad3  & 0,9   & -1443 & -2727 & -2044,8 & 407,361 \\
    zad3  & 1     & -725  & -2752 & -1786,2 & 694,7628 \\
    \end{tabular}%
  \label{tab:addlabel}%
  \caption{Badanie prawdopodobieństwa krzyżowania}
\end{table}%

\image{images/Badanie p. krzyżowania_1}{Badanie prawdopodobieństwa krzyżowania - problem 1}
\image{images/Badanie p. krzyżowania_2}{Badanie prawdopodobieństwa krzyżowania - problem 2}
\image{images/Badanie p. krzyżowania_3}{Badanie prawdopodobieństwa krzyżowania - problem 3}
\subsubsection{Wnioski}
Wybór prawdopodobieństwa krzyżowania nie ma znaczenia dla dwóch pierwszych problemów.
Wybrane zostaje prawdopodobieństwo krzyżowania "0.2", ponieważ zwraca ono najlepsze wyniki dla problemu 3.

\subsection{Prawdopodobieństwo mutacji}
\subsubsection{Cel badania}
Celem badania jest wybranie prawdopodobieństwa mutacji, dla którego algorytm genetyczny daje najlepsze wyniki.

\subsubsection{Stałe parametry badania}
% Table generated by Excel2LaTeX from sheet 'Badanie p. mutacji'
\begin{table}[htbp]
  \centering
    \begin{tabular}{rrrlr}
    \multicolumn{1}{p{4.215em}}{\textbf{Rozmiar populacji}} & \multicolumn{1}{p{4.215em}}{\textbf{Liczba pokoleń}} & \multicolumn{1}{p{4.215em}}{\textbf{Rozmiar turnieju}} & \multicolumn{1}{p{4.215em}}{\textbf{Operator selekcji}} & \multicolumn{1}{p{5.645em}}{\textbf{Pr. krzyżowania}} \\
    \midrule
    500   & 1500  & 3     & Turniej & 0,2 \\
    \end{tabular}%
  \caption{Stałe parametry badania}
  \label{tab:addlabel}%
\end{table}%

\subsubsection{Wyniki i wykresy}
% Table generated by Excel2LaTeX from sheet 'Badanie p. mutacji'
\begin{table}[htbp]
  \centering
    \begin{tabular}{crrrrr}
    \multicolumn{1}{p{4.215em}}{\textbf{Problem}} & \multicolumn{1}{p{5.57em}}{\textbf{Pr. mutacji}} & \multicolumn{1}{p{4.215em}}{\textbf{best}} & \multicolumn{1}{p{4.215em}}{\textbf{worst}} & \multicolumn{1}{p{4.215em}}{\textbf{avg}} & \multicolumn{1}{p{4.215em}}{\textbf{std}} \\
    \midrule
    zad1  & 0,01  & -1021 & -2393 & -1724,4 & 434,3214 \\
    zad1  & 0,05  & -340  & -1747 & -1315,2 & 554,7603 \\
    zad1  & 0,1   & -1025 & -2527 & -1479 & 599,7556 \\
    zad1  & 0,2   & -340  & -1814 & -921  & 546,9965 \\
    zad1  & 0,3   & -341  & -1065 & -898  & 279,1143 \\
    zad1  & 0,5   & -347  & -1120 & -639,4 & 352,29 \\
    zad1  & 0,8   & -319  & -1046 & -616,8 & 344,0834 \\
    zad1  & 0,9   & -319  & -1690 & -739,8 & 542,7045 \\
    zad1  & 1     & -348  & -1726 & -917  & 513,0902 \\
    \midrule
    zad2  & 0,01  & -1100 & -1692 & -1219,6 & 236,2114 \\
    zad2  & 0,05  & -1100 & -1100 & -1100 & 0 \\
    zad2  & 0,1   & -424  & -1100 & -964,8 & 270,4 \\
    zad2  & 0,2   & -424  & -1100 & -559,2 & 270,4 \\
    zad2  & 0,3   & -424  & -424  & -424  & 0 \\
    zad2  & 0,5   & -424  & -424  & -424  & 0 \\
    zad2  & 0,8   & -424  & -424  & -424  & 0 \\
    zad2  & 0,9   & -424  & -1036 & -913,6 & 244,8 \\
    zad2  & 1     & -424  & -1042 & -793,6 & 299,3417 \\
    \midrule
    zad3  & 0,01  & -2810 & -9189 & -5067 & 2269,493 \\
    zad3  & 0,05  & -2833 & -5655 & -3951,2 & 956,525 \\
    zad3  & 0,1   & -2122 & -8460 & -4458,4 & 2174,983 \\
    zad3  & 0,2   & -3398 & -7148 & -5231,2 & 1368,857 \\
    zad3  & 0,3   & -2215 & -5677 & -3717 & 1112,463 \\
    zad3  & 0,5   & -2110 & -3448 & -2882,4 & 496,1506 \\
    zad3  & 0,8   & -2116 & -3494 & -2920 & 519,8896 \\
    zad3  & 0,9   & -1429 & -4342 & -3266,8 & 1234,406 \\
    zad3  & 1     & -3528 & -4194 & -3929,4 & 316,8739 \\
    \end{tabular}%
  \caption{Badanie prawdopodobieństwa mutacji}
  \label{tab:addlabel}%
\end{table}%

\image{images/Badanie p. mutacji_1}{Badanie prawdopodobieństwa mutacji - problem 1}
\image{images/Badanie p. mutacji_2}{Badanie prawdopodobieństwa mutacji - problem 2}
\image{images/Badanie p. mutacji_3}{Badanie prawdopodobieństwa mutacji - problem 3}
\subsubsection{Wnioski}
Wybrane zostaje prawdopodobieństwo mutacji "0.9", ponieważ zwraca ono najlepsze wyniki dla wszystkich problemów.

\section{Porównanie algorytmu genetycznego z metodami "naiwnymi"}
Przy porównaniu uruchomiono 10 razy algorytm ewolucyjny (po 1500 generacji), 15000 razy metoda losowa,
oraz 30 razy metoda losowych mutacji (po 1500 generacji).
Metoda losowych mutacji polega na wylosowaniu osobnika i wykonywaniu na nim losowych mutacji, co jakiś czas cofając się do najlepszego znalezionego rozwiązania.

% Table generated by Excel2LaTeX from sheet 'Porównanie algorytmów'
\begin{table}[htbp]
  \centering
    \begin{tabular}{rrrlrr}
    \multicolumn{1}{p{4.215em}}{\textbf{Rozmiar populacji}} & \multicolumn{1}{p{4.215em}}{\textbf{Liczba pokoleń}} & \multicolumn{1}{p{4.215em}}{\textbf{Rozmiar turnieju}} & \multicolumn{1}{p{4.215em}}{\textbf{Operator selekcji}} & \multicolumn{1}{p{5.57em}}{\textbf{Pr. krzyżowania}} & \multicolumn{1}{p{4.215em}}{\textbf{Pr. mutacji}} \\
    \midrule
    500   & 1500  & 3     & Turniej & 0,2   & 0,9 \\
    \end{tabular}%
  \label{tab:addlabel}%
  \caption{Parametry algorytmu genetycznego}
\end{table}%

% Table generated by Excel2LaTeX from sheet 'Porównanie algorytmów'
\begin{table}[htbp]
  \centering
    \begin{tabular}{l|rrrr|rrrr|}
    \multicolumn{1}{c|}{\textbf{Instancja}} & \multicolumn{4}{c|}{\textbf{Algorytm ewolucyjny [10 x 1500]}} & \multicolumn{4}{c|}{\textbf{Metoda losowa [15000]}} \\
          & \multicolumn{1}{c}{\textbf{best}} & \multicolumn{1}{c}{\textbf{worst}} & \multicolumn{1}{c}{\textbf{avg}} & \multicolumn{1}{c|}{\textbf{std}} & \multicolumn{1}{c}{\textbf{best}} & \multicolumn{1}{c}{\textbf{worst}} & \multicolumn{1}{c}{\textbf{avg}} & \multicolumn{1}{c|}{\textbf{std}} \\
    \midrule
    zad1  & -341  & -2527 & -1110,6 & 742,775 & -270894 & -368663 & -277311 & 10502,92 \\
    zad2  & -424  & -1036 & -731,2 & 304,8071 & -135953 & -285244 & -163922 & 12024,88 \\
    zad3  & -2059 & -4213 & -2985,5 & 620,792 & -478885 & -686348 & -510788 & 24221,65 \\
    \end{tabular}%
  \label{tab:addlabel}%
\end{table}%
% Table generated by Excel2LaTeX from sheet 'Porównanie algorytmów'
\begin{table}[htbp]
  \centering
    \begin{tabular}{|rrrr}
    \multicolumn{4}{|c}{\textbf{Metoda losowych mutacji [30 x 1500]}} \\
    \multicolumn{1}{|c}{\textbf{best}} & \multicolumn{1}{c}{\textbf{worst}} & \multicolumn{1}{c}{\textbf{avg}} & \multicolumn{1}{c}{\textbf{std}} \\
    \midrule
    -349442 & -604234 & -470676 & 57255,29 \\
    -233683 & -420265 & -330202 & 50262,56 \\
    -637791 & -935288 & -806516 & 79839,92 \\
    \end{tabular}%
  \caption{Porównanie algorytmu genetycznego z naiwnymi}
  \label{tab:addlabel}%
\end{table}%



\image{images/Porównanie algorytmów_1}{Porównanie algorytmów - problem 1}
\image{images/Porównanie algorytmów_2}{Porównanie algorytmów - problem 2}
\image{images/Porównanie algorytmów_3}{Porównanie algorytmów - problem 3}

\section{Podsumowanie}
Algorytm genetyczny pozwala na znaczące poprawy wyników dla danego problemu.
\image{images/solution1}{Najlepsze znalezione rozwiązanie do problemu 1}
\image{images/solution2}{Najlepsze znalezione rozwiązanie do problemu 2}
\image{images/solution3}{Najlepsze znalezione rozwiązanie do problemu 3}

\end{document}
